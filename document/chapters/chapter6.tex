\chapter{Conclusion and Future Work}%

% \begin{introduction}
%     something
% % A short description of the chapter.
% % A memorable quote can also be used.
% \end{introduction}


\section{Conclusion}

This dissertation has explored the development of an Mixed-Reality-assisted, Digital Twin-enabled robot collaborative system with human-in-the-loop control. The primary objective consisted on enhancing Human-Robot Collaboration by integrating advanced technologies that bridge the gap between physical and virtual environments supporting remote collaboration and, thereby, fostering more efficient and intuitive interactions in manufacturing settings.

\textbf{On-Site interaction} was a crucial aspect of the project. By utilizing Handheld Devices such as tablets and smartphones, on-site participants were able to share live views of their surroundings. Furthermore, augmented reality elements were layed upon the mixed-reality application to provide visual and audio cues to the operator and enhance the robot's movements awareness. 

In terms of \textbf{remote visualization and interaction}, a foundational 2D interface was accessible via standard devices like laptops. This interface enabled remote participants to visualize the collaboration scenario and understand the task context effectively. Furthermore, the system's capabilities allow remote operation of the robot through the \ac{MR} application, given that bi-directional communication was implemented. This enhancement empowered remote users to interact with and manipulate the collaborative environment, bringing them closer to the on-site experience and improving overall collaboration efficacy.


% The camera mounted on top of the robot also allowed to share live-feed of the workspace, therefore successfully enabling dynamic and real-time collaboration with the robot within the designated environment. 
% This approach allowed remote collaborators to gain a comprehensive understanding of the collaborative space, enhancing situational awareness and facilitating better coordination.

Regarding \textbf{automation and immersion}, a camera was mounted on the robot to automate the process of environment sharing with remote participants. This feature relieved on-site participants from the responsibility of manually sharing visual information, as the robot could now autonomously provide live feeds of the workspace. This automation not only improved efficiency but also enhanced the immersive experience for remote users by offering real-time visual insights into the operational environment.

Throughout the development process, we leveraged the Unity game engine for robot model development and employed the Robot Operating System for seamless communication between the physical robot and its digital twin. The use of Vuforia facilitated precise pose registration, ensuring accurate alignment between virtual and physical models. By integrating both visual and audio cues as well as intuitive controls within the shared environment, we enhanced user awareness of the robot's movements and provided a user-friendly interface for robot manipulation.

In conclusion, the project successfully met its defined objectives by creating a system that enhances Human-Robot Collaboration through advanced Mixed-Reality and Digital Twin technologies, focusing on the remote member capabilities. Integrating these technologies resulted in a more intuitive, efficient, and safe collaborative environment, aligning with the core values of Industry 5.0. This proposed system demonstrates significant potential for improving manufacturing processes by combining human expertise with robotic precision, ultimately contributing to more flexible and human-centric industrial practices.

\section{Future Work}

Despite having achieved its primary goals, there are several areas for future exploration to further enhance the system's capabilities and impact.

\textbf{Conducting User Experience Research and Evaluation} is a vital next step to refine the system based on user feedback and performance metrics. By performing usability studies, we can identify pain points and areas for improvement, ensuring that the system meets the needs of its users effectively. Analyzing task performance data will help optimize workflows and enhance overall efficiency.

Future work related to this includes:

\begin{itemize}
    \item \textbf{Longitudinal Studies}: Assessing the long-term impact of the technology on collaboration efficiency will provide insights into how the system influences productivity over time. This can reveal trends and patterns that inform further enhancements.
    \item \textbf{Cross-Cultural Evaluations}: Studying how users from different cultural backgrounds interact with the system will ensure its applicability and accessibility in diverse settings. This evaluation can help tailor the system to accommodate varying user preferences and communication styles.
    \item \textbf{Ergonomic Assessments}: Ensuring that prolonged use of AR devices does not cause discomfort or health issues is essential for user well-being. Conducting ergonomic studies will help optimize device usage and interface design to promote comfort and reduce fatigue.
\end{itemize}

In addition to user experience research, future work could focus on:

\begin{itemize} 
    \item \textbf{Enhancing Immersive Technologies}: Exploring the potential of advanced devices like the Microsoft HoloLens 2 can further enhance the immersive experience of remote collaboration. Integrating mixed reality headsets can provide users with more natural and intuitive interactions within the collaborative environment.
    \item \textbf{Improving Communication Tools}: Integrating information-sharing tools such as voice communication and real-time annotations will facilitate more effective collaboration between on-site and remote participants. This enhancement can lead to better understanding, quicker decision-making, and a more cohesive teamwork experience.
    \item \textbf{Advanced Interaction Modalities}: Incorporating gesture recognition and voice commands can make the system more accessible and reduce reliance on manual input devices. These modalities can provide a more intuitive control mechanism, especially in environments where traditional input devices are impractical.
    \item \textbf{System Performance Optimization}: Addressing challenges related to network latency and positioning accuracy will improve the system's responsiveness and reliability. Implementing advanced communication protocols and optimizing data processing can enhance real-time interactions.
    \item \textbf{Security and Privacy Measures}: Strengthening encryption and authentication protocols will protect user data and ensure secure communication channels. This is crucial for maintaining trust and compliance with data protection regulations.
\end{itemize}

By pursuing these future developments, the system can significantly improve its effectiveness and user satisfaction. Continuous refinement based on user feedback and technological advancements will contribute to its adoption in various industrial contexts, ultimately enhancing human-robot collaboration and advancing the principles of Industry 5.0.

\subsection{Final Remarks}

This dissertation has laid the groundwork for an innovative approach to human-robot collaboration in manufacturing environments. By integrating AR and DT technologies, we have demonstrated the potential for creating systems that are not only efficient but also human-centric. The fusion of human intuition with robotic capabilities opens new avenues for productivity and innovation.

The journey does not end here. The insights gained and the foundation established through this work pave the way for future explorations that can further bridge the gap between humans and machines. Embracing continuous improvement and adaptation will ensure that such systems remain relevant and impactful in the ever-evolving landscape of industrial automation.