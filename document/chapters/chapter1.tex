\chapter{Introduction}%
\label{chapter:introduction}

% \begin{introduction}
% % A short description of the chapter.
% % A memorable quote can also be used.
% This chapter aims to describe the context of this dissertation, outlining the primary goals and contributions of the developed work. 
% Additionally, it will present the structure of this document.
% \end{introduction}

\section{Motivation}
\label{section:Motivation}
% \input{chapters/introduction/motivation}
The First Industrial Revolution, powered by steam engines, paved the way for subsequent revolutions driven by electricity, automation, machinery, 
and the internet. Each revolution introduced groundbreaking technologies that reshaped industries, emphasizing the companies' need to prioritize reskilling 
and upskilling their workforce.

In recent years, Industry 4.0 has marked a paradigm shift toward the digitization of manufacturing processes. By integrating technologies such as the \ac{IoT}, \ac{AI}, and automation, Industry 4.0 has fostered highly interconnected and intelligent manufacturing ecosystems. These advancements have led to a notable decrease in the reliance on human labor, as automation takes center stage in many operations.
However, as factories become increasingly autonomous, the unique cognitive and adaptive capabilities that humans bring to complex decision-making, creativity, and problem-solving remain irreplaceable, and experts argue that complete removal of humans from the manufacturing processes is not feasible. Instead, there is a growing emphasis on fostering collaborative partnerships between humans and intelligent machinery \cite{Weiss2021}.

As collaborative environments evolve, robots have become indispensable in various domains, leading to increased complexity in these scenarios. Therefore, advanced solutions are needed to enhance \ac{HRC}. For instance, in flexible manufacturing systems, robots must adapt to frequent changes in production tasks, requiring advanced cognitive capabilities to handle non-repetitive operations. Additionally, in collaborative assembly lines, robots need to interact safely and efficiently with human workers, often in unstructured environments where unpredictable human movements pose significant safety and operational challenges. One promising approach consists on integrating Mixed Reality (\ac{MR}) technologies as a medium for collaboration, encompassing \ac{VR} for the remote user and Augmented Reality (\ac{AR}) for the on-site one, by blending the physical and digital world. By providing immersive experiences that transcend traditional reality and overcome geographical constraints, this relationship enables real-time collaboration among individuals from different locations \cite{whatismixedreality}.
* TODO: add more references - por 3/4 aqui - verificar melhor ao longo do resto do texto, 4 que façam sentido

However, the potential of \ac{MR} to enhance remote collaboration is currently hindered by several critical limitations. These include, not only limited perspective and context capture, which impede remote collaborators' understanding and decision-making capabilities, as well as a lack of multisensory data collection, restricting comprehensive environmental comprehension. 
Additionally, \ac{MR} interaction with physical objects often lacks the precision required for detailed tasks, particularly in dynamic scenarios. These challenges diminish the effectiveness of \ac{MR} in facilitating thorough context sharing and impact the overall efficiency and safety of 
collaborative tasks.
* TODO: add more references - verificar melhor ao longo do resto do texto algumas que façam sentido aqui

\FloatBarrier

\section{Goals}
\label{section:Goals}
% \input{chapters/introduction/goals}
% The primary goal of this dissertation consists on leveraging (\ac{MR}) alongside a static robotic arm (UR10e) to support remote collaboration scenarios. 

The primary goal of this dissertation is to enhance remote collaboration between human operators by utilizing a robotic arm (UR10e) and \ac{MR} technologies. This framework enables dynamic and immersive collaboration, where both on-site and remote participants can interact with both the robot and the shared environment in real time. 

% In order to properly achieve this, the main goal can be broken down into the following objectives:
According to the collaborative element being addressed, namely: 
\begin{itemize}
    \item \textbf{On-Site Interaction:}
    \begin{itemize}
        \item Enable dynamic and real-time robot manipulation through \ac{AR} within the designated environment.
        \item Infer robot state
        \item Visualize and interact with the robot's workspace through \ac{AR} safety-zones, enhancing the user's situational awareness.
        \item Utilize \ac{HHDs}, such as tablets or smartphones, to share live views of the surroundings, allowing remote collaborators 
        to gain a comprehensive understanding of the collaborative space.
        % \item Explore the potential of using advanced technologies like the Hololens 2 to enhance the immersive experience of remote collaboration. 
        % - deixo estar?????
    \end{itemize}
    \item \textbf{Remote Visualization and Interaction:}
    \begin{itemize}
        \item Provide remote member with a foundational \ac{UI} interface, such as a laptop screen, to visualize the collaboration scenario and 
        task context.
        \item Establish a bidirectional communication, enabling remote operation of the robot arm via the \ac{MR} application, enhancing the user's ability to interact with the on-site physical space of his/her counterpart.
        % \item Investigate the use of \ac{VR} headsets, such as the Oculus Quest, for a more immersive remote collaboration experience. - deixo estar?????
    \end{itemize}
\end{itemize}


\section{Thesis Structure}
% \input{chapters/introduction/structure}
* TODO: confirmar se sao efetivamente 6 capitulos 
This dissertation is structured across six chapters, each representing a logical progression in the development of this work within the context of \ac{HRC} and Industry 5.0.
%  The following sections provide a comprehensive overview of each chapter and its contribution to the overall thesis.

The first chapter introduces the project by framing the motivation behind this work as well as defining the baseline for this dissertation's development. The following chapter presents a thorough state-of-the-art review, with a focus on key concepts such as Industry 5.0, \ac{HRC}, Collaborative Robots, \ac{DRs}, and \ac{DTs}. It also includes an analysis of two case studies involving \ac{AR}-\ac{DT} solutions in industrial settings. The third chapter discusses the various implementation tools utilized in the project development, such as Unity, Vuforia, and ROS.
Chapter four outlines the framework established for the project and details the specific features implemented, distinguishing those meant for remote members from those designed for on-site collaborators. The fifth chapter is devoted to evaluating the developed features, identifying their limitations, and presenting an application example where the solution could be effectively implemented. Finally, the sixth chapter concludes the thesis by summarizing the project's achievements and suggesting possible directions for future work.

% Chapter 1: Motivation and Goals
% The introductory chapter frames the motivation behind this work, underscoring the importance of Human-Robot Collaboration (HRC) as a pivotal element of Industry 5.0. Unlike Industry 4.0, which prioritized automation and robotic autonomy, Industry 5.0 aims to harmoniously reintegrate humans into industrial systems to create symbiotic collaborations with machines. This shift arises from a need to augment human cognitive and physical abilities while maintaining the creativity and adaptability unique to human workers. This motivation is rooted in the growing need to improve flexibility, efficiency, and human-centric innovation within industrial environments.

% In this context, emerging technologies—such as collaborative robots (cobots), digital twins (DTs), and immersive digital realities—are critical enablers. Chapter 1 delineates how these technologies can serve as foundational components of a new paradigm where humans and robots collaborate effectively, supported by MR interfaces that bridge physical and digital worlds.

% The main objective of this dissertation is clearly outlined: to explore the potential of mixed reality in augmenting human-robot collaboration by integrating MR with a static robotic arm to support remote collaboration scenarios. The primary goal is further broken down into smaller, more specific objectives that address technical, functional, and experiential aspects of the integration, such as the development of real-time robot control interfaces, remote visualization of robot states, and the bidirectional communication between a digital twin and a physical robotic system.

% Chapter 2: State of the Art Review
% Chapter 2 presents a comprehensive review of the relevant literature and state-of-the-art technologies that inform this dissertation's development. The chapter is divided into key thematic areas:

% Industry 5.0 and its Key Drivers: This section traces the evolutionary path from Industry 4.0 to Industry 5.0, focusing on the reintroduction of the human element into automated systems. Particular emphasis is placed on the technological enablers, such as AI, collaborative robots (cobots), digital twins, and MR, that make this transition possible. These technologies are examined not only for their technical capabilities but also for their role in fostering a more human-centric industrial environment.

% Human-Robot Collaboration (HRC): This section provides an in-depth exploration of HRC, its applications in Industry 5.0, and how it differs from traditional automation paradigms. Collaborative robots and their increasing role in industrial settings are discussed, particularly in light of how mixed reality can enhance interaction between humans and robots in shared workspaces.

% Digital Realities (AR, VR, MR, XR): Here, the dissertation delves into the different facets of digital realities, clarifying the conceptual distinctions between augmented reality (AR), virtual reality (VR), mixed reality (MR), and extended reality (XR). Since MR lacks a universally agreed-upon definition, a working definition is established for the scope of this dissertation, where MR is characterized by the seamless integration of virtual and physical environments for real-time interaction between human operators and robots.

% Digital Twins: Digital twins are explored in their role as virtual representations of physical systems, providing real-time data and feedback loops that enable enhanced monitoring, control, and prediction of robot behavior. The integration of digital twins in industrial applications is examined, focusing on their impact on improving remote collaboration efficiency and system flexibility.

% Summary: The chapter concludes with a synthesis of these technologies, highlighting their interconnectedness and potential synergies in supporting remote human-robot collaboration in industrial settings.

% Chapter 3: Methodology
% The third chapter presents the methodology and design approach for the project. The focus is on outlining the technical workflow for integrating the static robotic arm into a mixed reality environment. The chapter begins by acknowledging the availability of a real robotic system (UR10e) provided by the project supervisors and proceeds to elaborate on the development pipeline.

% The methodology is structured around key phases:

% Robot Model Integration: The process of importing the UR10e robotic arm into the Unity environment is detailed, including the creation of a digital twin (DT) that mirrors the robot’s real-world movements.

% Pose Registration: A critical component of the methodology, pose registration ensures that the digital robot aligns perfectly with the physical robot’s position in space. This phase requires careful calibration to maintain spatial accuracy during operations.

% Bidirectional Communication: The development of communication channels between the Unity-based digital twin and the ROS environment is elaborated, with emphasis on message-passing mechanisms that allow data to flow between the virtual and physical systems.

% ROS Development: Key ROS nodes were developed to facilitate joint state updates and real-time control of the robot through the mixed reality interface. These nodes are responsible for handling position data, joint states, and environmental information, ensuring synchronization between the digital twin and the physical robot.

% Chapter 4: Implementation of Features
% Chapter 4 focuses on the concrete implementation of the system’s features, targeting both the on-site and remote collaborators:

% On-Site Collaboration: This section describes the tools developed to allow local users to interact with the robot. This includes dynamic robot control through handheld devices (e.g., tablets), as well as augmented visualization of the robot’s workspace using MR technologies.

% Remote Visualization and Interaction: For remote participants, a set of tools were developed to enable real-time visualization and manipulation of the robotic system. This includes live camera feeds from the robot’s perspective, augmented by MR overlays, and the ability to remotely control the robot using a 2D interface or through an MR application.

% The chapter illustrates how these features enhance collaboration by improving situational awareness, minimizing communication delays, and creating a more immersive and intuitive collaborative environment.

% Chapter 5: Challenges and Limitations
% In this chapter, the technical and practical challenges encountered during the project are discussed. Special attention is given to the lack of comprehensive documentation and existing frameworks for integrating mixed reality systems with ROS-based robotics. Issues such as network latency, data transmission limitations, and hardware-software compatibility are explored, alongside the strategies employed to mitigate these challenges.

% Chapter 6: Conclusions and Future Work
% The final chapter provides a summary of the project's accomplishments, reflecting on how the initial goals were met and how the developed system succeeded in enhancing human-robot collaboration in a mixed reality environment. The chapter emphasizes the broader applicability of the implemented solution and its potential to scale to more complex industrial scenarios.

% Finally, the chapter identifies areas for future research and development, such as:

% Extending the MR interface to accommodate more sophisticated task management tools.
% Enhancing the communication protocols to reduce latency in real-time robot manipulation.
% Investigating the use of more advanced AR/VR devices, such as the HoloLens 2 and Oculus Quest, to further enhance remote collaboration.
% Conducting longitudinal user studies to assess the system’s long-term impact on collaboration efficiency and user satisfaction.
