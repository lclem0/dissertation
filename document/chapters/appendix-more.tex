\chapter{Additional content}


\section{Unity Robotics Hub Overview}

    Before diving into the specifics of establishing the ROS-Unity connection and further develop the project, both tutorials and resources available 
    in the Unity Robotics Hub were studied. This GitHub repository serves as a central hub for tools, tutorials, and documentation tailored for robotic 
    simulation in Unity.

    \subsection{Available Documentation}

    It offers a range of tutorials that are invaluable for setting up and extending ROS-Unity integration, as well as to understand how ROS concepts work inside Unity's environment:
    
    \begin{itemize}
        \item \textbf{ROS–Unity Integration: Initial Setup} - Guides you through the initial steps of setting up communication between ROS and Unity, including package installation and network configuration.
        
        \item \textbf{ROS–Unity Integration: Network Description} - Provides a detailed overview of network settings and offers troubleshooting tips for connectivity issues.
        
        \item \textbf{ROS–Unity Integration: Publisher} - Teaches you how to publish messages from a Unity scene to ROS, with practical examples involving GameObject data.
        
        \item \textbf{ROS–Unity Integration: Subscriber} - Demonstrates how to subscribe to ROS topics in Unity and use the received messages to alter objects in a Unity scene.
        
        \item \textbf{ROS–Unity Integration: Unity Service} - Covers the implementation of ROS services within Unity, allowing Unity to respond to ROS service requests.
        
        \item \textbf{ROS–Unity Integration: Service Call} - Explains how to call external ROS services from Unity, enabling Unity to request data or actions from ROS nodes.
    \end{itemize}
    
    The repository also includes example scripts that correspond to each tutorial.
    
    \section{Establishing the Network Connection}
    After reviewing the Unity Robotics Hub tutorial on network integration, it became clear that establishing a network connection between the Unity and ROS environments was the first crucial step in remote application development. The process involves:
    \begin{itemize}
        \item \textbf{Setting up the network:} Connect the Unity laptop to a Wi-Fi network, then connect the Ubuntu laptop running ROS to the hotspot created by the Unity laptop.
        \item \textbf{Configuring the IP address:} Use the IP address from the Unity laptop within the Unity inspector as shown in Figure \ref{fig:unity_connection}, and in the \texttt{ROSConnection.cs} script to ensure proper communication.
        \item \textbf{Specify the IP Address in ROS Workspace:} A new \textit{.launch} file was created to initialize new nodes, including the \texttt{server\_endpoint} node from the \texttt{ros\_tcp\_endpoint} package, crucial for establishing a proper connection between the ROS and Unity environments. An example of the IP definition in the \textit{.launch} file can be seen below.
    \end{itemize}
    
    \begin{figure}[htbp]
        \centering
        \includegraphics[width=0.5\linewidth]{figs/connection_inspector.png}
        \caption{Unity Connection Inspector}
        \label{fig:unity_connection}
    \end{figure}
    
    % \begin{figure}[htbp]
    %     \centering
    %     \includegraphics[width=\linewidth]{tcp_ip_connection_launch.jpg}
    %     \caption{TCP/IP ROS Node Connection Setup}
    %     \label{fig:tcp_ip_connection}
    % \end{figure}
    \begin{verbatim}
            <arg name="tcp_ip" default="192.168.137.202"/>
            <arg name="tcp_port" default="10000"/>
            
            <node name="server_endpoint" pkg="ros_tcp_endpoint" 
                type="default_server_endpoint.py" args="--wait" output="screen" 
                respawn="true">
                <param name="tcp_ip" type="string" value="$(arg tcp_ip)"/>
                <param name="tcp_port" type="int" value="$(arg tcp_port)"/>
            </node>
    \end{verbatim}
    
    
    This setup is fundamental for the Unity environment to interact effectively with ROS, allowing for real-time data exchange and control commands to be sent between the two systems. Further details are available in the Unity Robotics Hub tutorial on \href{https://github.com/Unity-Technologies/Unity-Robotics-Hub/blob/main/tutorials/ros_unity_integration/network.md}{ROS-Unity integration}.