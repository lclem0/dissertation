\chapter{Introduction}%
\label{chapter:introduction}

\begin{introduction}
% A short description of the chapter.
% A memorable quote can also be used.
This chapter aims to describe the context of this dissertation, outlining the primary goals and contributions of the developed work. 
Additionally, it will present the structure of this document.
\end{introduction}

\section{Motivation}
\label{section:Motivation}
% \input{chapters/introduction/motivation}
The First Industrial Revolution, powered by steam engines, paved the way for subsequent revolutions driven by electricity, automation, machinery, 
and the internet. Each revolution introduced groundbreaking technologies that reshaped industries, necessitating companies to prioritize reskilling 
and upskilling their workforce.

However, experts argue that complete automation, eliminating humans from manufacturing processes, is not feasible \cite{Weiss2021}. Instead, there is 
a growing emphasis on fostering collaborative partnerships between humans and intelligent machinery.

As collaborative environments evolve, robots have become indispensable in various domains.
% including the automation of medical services \cite{health_ref} and warehouse management \cite{warehouse_ref}.
The complexity of these scenarios necessitates advanced solutions to enhance the \ac{HRI}. One promising approach is the integration of \ac{XR} 
technologies, which encompass \ac{VR}, \ac{AR}, and \ac{MR}. \ac{XR} technologies blend the physical and digital worlds, providing immersive 
experiences that transcend traditional reality and overcome geographical constraints, enabling real-time collaboration among individuals from 
different locations.

However, the potential of \ac{XR} to enhance remote collaboration is currently hindered by several critical limitations. These include limited 
perspective and context capture, which impede remote collaborators' understanding and decision-making capabilities, and a lack of multisensory 
data collection, restricting comprehensive environmental comprehension. 
Additionally, \ac{XR}'s interaction with physical objects often lacks the precision required for detailed tasks, particularly in dynamic scenarios. 
These challenges diminish the effectiveness of \ac{XR} in facilitating thorough context sharing and impact the overall efficiency and safety of 
collaborative tasks.

\section{Goals}
% \label{section:Goals}
% \input{chapters/introduction/goals}
The primary goal of this dissertation is to leverage \ac{MR} alongside a static robotic arm (UR10e) to support remote collaboration scenarios. 
This involves transforming human-robotic collaboration by integrating \ac{XR} technologies and robotic capabilities to enhance both on-site and 
remote collaboration experiences.

To achieve this, the dissertation proposes the following objectives:

\begin{itemize}
    \item \textbf{On-Site Interaction:}
    \begin{itemize}
        \item Enable dynamic and real-time collaboration with the robot within the designated environment.
        \item Utilize handheld devices (HHDs), such as tablets or smartphones, to share live views of the surroundings, allowing remote collaborators 
        to gain a comprehensive understanding of the collaborative space.
        % \item Explore the potential of using advanced technologies like the Hololens 2 to enhance the immersive experience of remote collaboration. 
        % - deixo estar?????
    \end{itemize}
    \item \textbf{Remote Visualization and Interaction:}
    \begin{itemize}
        \item Provide remote participants with a foundational 2D interface, such as a laptop screen, to visualize the collaboration scenario and 
        task context.
        \item Develop the capability for remote operation of the robot via the \ac{XR} application, enhancing the remote participant's ability to 
        interact and manipulate the collaborative environment.
        % \item Investigate the use of \ac{VR} headsets, such as the Oculus Quest, for a more immersive remote collaboration experience. - deixo estar?????
    \end{itemize}
    \item \textbf{Automation and Immersion:}
    \begin{itemize}
        \item Integrate a camera into the robot to automate the process of environment sharing with remote participants, assisting on-site participants 
        by delegating visual sharing to the robot.
        % \item Incorporate headsets for environmental visualization to enhance the remote participant's perception and interaction with the collaborative 
        % space. - deixo estar?????
    \end{itemize}
    % \item \textbf{Communication Efficiency:}
    % \begin{itemize}
    %     \item Improve communication between on-site and remote members by integrating information-sharing tools, such as voice and annotations, 
    %     to facilitate more effective collaboration. - deixo estar?????
    % \end{itemize}
\end{itemize}

By addressing these objectives, this dissertation aims to create a robust framework that enhances remote collaboration through the innovative use 
of \ac{MR} and robotic technologies.



\section{Thesis' Structure}
% \input{chapters/introduction/structure}
fazer!!!!! no fim 